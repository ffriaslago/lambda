\documentclass{beamer}    % [aspectratio=169]
\pagestyle{empty}

\usepackage{amsmath}
\usepackage{mathspec}
\usepackage{nicefrac}
%fontspec

\definecolor{morado}{RGB}{85, 39, 107}
\definecolor{naranja_muy}{RGB}{178, 104, 32}
\definecolor{naranja_osc}{RGB}{255, 191, 128}
\definecolor{naranja_claro}{RGB}{255, 229, 204}
\definecolor{naranja_ter}{RGB}{178, 104, 32}
\definecolor{naranja_var}{RGB}{178, 104, 32}
\definecolor{fondo}{RGB}{255, 255, 255}
\definecolor{frente}{RGB}{0, 0, 0}
\definecolor{color_r}{RGB}{164, 12, 0}
\definecolor{color_a}{RGB}{10, 131, 166}
\definecolor{casilla}{RGB}{237, 168, 100}
\definecolor{col_b}{RGB}{128, 175, 253}    % predicados
\definecolor{col_c}{RGB}{246, 0, 10}       % funciones
\definecolor{col_d}{RGB}{255, 156, 59}     % variables

\def\cxb{\color{col_b}}
\def\cxc{\color{col_c}}
\def\cxd{\color{col_d}}

\def\sufijo{_blanco}

\usetheme{default}
\setbeamercolor{background canvas}{bg=fondo}
\setbeamercolor{normal text}{fg=frente}
\setbeamercolor{title}{fg=frente}
\setbeamercolor{frametitle}{bg=naranja_osc!50!black, fg=black!10}
\setbeamercolor{block body}{bg=naranja_osc!50!black}
\setbeamercolor{block title}{bg=yellow}

\setbeamertemplate{blocks}[rounded]

\usefonttheme{professionalfonts}
% \setsansfont[SmallCapsFont={Linux Biolinum Keyboard O},ItalicFont={Linux Libertine O/I}]%
\setsansfont[SmallCapsFont={EB Garamond SC},ItalicFont={Linux Libertine O/I},BoldFont={Linux Libertine O/B}]%
{Aroania}
%{Linux Biolinum O}
%{Cantarell}
%{Droid Sans} % Tal vez demasiado apretada
%{Liberation Sans}
%{Luxi Sans}

\setmonofont[BoldFont={TerminusTTF-Bold}]{Terminus (TTF)}
\setmathsfont(Digits,Latin,Greek){Neo Euler}

\usepackage{amssymb}
\usepackage{pagecolor}
\usepackage{pbox}
\usepackage{graphicx}
\usepackage{fancyvrb}

\setlength\parindent{0pt}

\newcommand\id{\mbox{indeg}}
\newcommand\od{\mbox{outdeg}}

\newcommand\N{\mathbb{N}}
\newcommand\Z{\mathbb{Z}}
\newcommand\R{\mathbb{R}}

\newcommand\ret{↵\ } % ⏎, 〈RET〉

\newcommand\tr{^{\mbox{\footnotesize t}}}   % transpuesta

\newcommand\inp[1]{\colorbox{gray!40}{#1}}

\usepackage{stmaryrd}       % flecha

% \usepackage[colorlinks=true, urlcolor=blue]{hyperref}
\usepackage{hyperref}

\usepackage{array}
% alineada a la derecha, anchura fija
\newcolumntype{R}[1]{>{\raggedright\arraybackslash}m{#1}}
% mat., alineada a la derecha, anchura fija
\newcolumntype{O}[1]{>{\hfill\arraybackslash$}m{#1}<{$}}
% mat., alineada a la izquierda, anchura fija
\newcolumntype{I}[1]{>{\arraybackslash$}m{#1}<{$}}
\newcolumntype{C}{>{$}c<{$}}
\newcolumntype{M}[1]
{>{\centering\arraybackslash $}m{#1}<{$}}

\newcommand\figs{../figs}
\newcommand\cod{../scripts}

\newcommand\espacio{␣}
\newcommand\vacio{\mbox{\fontspec{FreeSerif}∅}}
\newcommand\vacin{\mbox{\fontspec{FreeSerif}\footnotesize ∅}}

\pagecolor{morado!10} % 238, 233, 240 (.933, .914, .941)

\def\truca
{~\par}
\def\trucavuelve
{~\par\vspace{-\baselineskip}}

\usepackage[spanish]{babel}
\usepackage{colortbl}

\newcommand\gr{\color{gray}}

\usepackage{diagbox}
\usepackage[absolute,overlay]{textpos}

% Caracteres
%% \def\eqsem{\mbox{\fontspec{FreeSerif} ⧦}}
\def\true{\mbox{\fontspec{FreeSerif} ⊤}}
\def\false{\mbox{\fontspec{FreeSerif} ⊥}}

% Letras
%% \newcommand\true{\texttt{true}}
%% \newcommand\false{\texttt{false}}

% mathabx
\usepackage{mathabx}
\def\eqsem{\vDash\!\!\Dashv}
%% \def\true{\top}
%% \def\false{\perp}

% MnSymbol
%% \usepackage{MnSymbol}
%% \def\eqsem{\vDash\!\!\leftmodels}
%% \newcommand\true{\downvdash}
%% \newcommand\false{\perp}

\def\sa{{\color{naranja_muy} \bullet}}
\def\sb{{\color{blue!60!white} \star}}

\def\var#1{{\color{naranja_muy}#1}}
\def\fun#1{\texttt{#1}}
\def\funr#1{{\color{red}\texttt{#1}}}
\def\pred#1{{\color{blue!20}#1}}
\def\dom{{\cal D}}

\usepackage{tikz}

\usetikzlibrary{shapes.misc}

\tikzstyle{caja}=[]
\tikzstyle{caja_b}=[anchor=east]
\tikzstyle{caja_d}=[anchor=west]
\tikzstyle{caja_c}=%
 [anchor=east, text=gray, cross out, draw=red] % strike out

\def\clp{{\color{gray!80!naranja_muy}\texttt{CLIPS>}}}
\def\qry{{\color{gray!80!naranja_muy}\texttt{?-}}}
\def\bl#1{{\color{black!50!blue}\texttt{#1}}}
\def\ro#1{{\color{red!80!black}\texttt{#1}}}

\begin{document}
\shorthandoff{>}\shorthandoff{<}
\begin{frame}
\centerline{\emph{Cláusulas de Horn.}}
\centerline{\emph{Resolución SLD}}
\end{frame}

\begin{frame}
 \frametitle{Cláusulas de Horn}

 La estrategia común para aligerar las dificultades
 computacionales asociadas al uso del lenguaje de primer
 orden consiste en restringir el lenguaje,
 \vspace{5mm}

 buscando un compromiso entre expresividad y tratabilidad.
\end{frame}

\begin{frame}[t]
 \frametitle{Cláusulas de Horn}

 La restricción del lenguaje de primer orden típica de la
 programación lógica consiste en considerar exclusivamente
 cláusulas como estas:\vspace{5mm}

\begin{center}
\begin{tabular}{r@{\hspace{1cm}}l}
$p_1 \land \cdots \land p_n \rightarrow\phantom{\neg} q$ &
(\emph{positivas} o \emph{definidas})\\
\only<2->{
$\neg(p_1 \land \cdots \land p_n) \lor \phantom{\neg} q$ &\\
$\neg p_1 \lor \cdots \lor \neg p_n \lor \phantom{\neg} q$ &\\[3mm] \hline
&\\[1mm]
}
$p_1 \land \cdots \land p_n \rightarrow \neg q$ &
(\emph{negativas}).\\
\only<2->{
$\neg(p_1 \land \cdots \land p_n) \lor \neg q$ &\\
$\neg p_1 \lor \cdots \lor \neg p_n \lor \neg q$ &\\
}
\end{tabular}
\end{center}


\begin{itemize}
\only<1>{
\item Los símbolos $p_i$ y $q$ no representan aquí
  proposiciones únicamente, sino que pueden ser predicados
  con argumentos cualesquiera (\emph{átomos}).
\item Los cuantificadores se han eliminado (suponiendo
  implícitos los universales y recurriendo a funciones de
  Skolem para los existenciales).}
\only<2->{
\item Una cláusula de Horn es una en la que todas las
  literales son negaciones, salvo quizás una.
\item {\sc Alfred Horn} [1951]}
\end{itemize}
\end{frame}

\begin{frame}
 \frametitle{Resolución con cláusulas de Horn}

 El conjunto de cláusulas de Horn es cerrado para el cálculo
 de resolventes.\vspace{5mm}

 Para que dos cláusulas de Horn den lugar a una resolvente,
 al menos una de ellas ha de ser positiva:

 \begin{columns}[t]
 \begin{column}{.47\linewidth}
\[\begin{array}{r@{\quad}c}
\neg\sa \lor \cdots \lor \neg\sa \lor\phantom{\neg} q& +\\
\neg\sa \lor \cdots \lor \neg\sa \lor\neg q& -\\ \hline
& -\\
\end{array}\]
 \end{column}
 \begin{column}{.5\linewidth}
\[\begin{array}{r@{\quad}c}
\neg\sa \lor \cdots \lor \neg\sa \lor \neg\sa
\lor\phantom{\neg} q& +\\
\neg\sa \lor \cdots \lor \neg\sa \lor\phantom{\neg} p
\lor\neg\sa& +\\ \hline
& +\\
\end{array}\]
 \end{column}
 \end{columns}
\vspace{5mm}

\pause

El tipo de la otra cláusula coincide con el de la
resolvente.
\end{frame}

\begin{frame}
 \frametitle{Resolución con cláusulas de Horn}

 Supongamos que
 \begin{itemize}
 \item $S$ es un conjunto de cláusulas de Horn,
 \item $c$ es una cláusula de Horn negativa (por
  ejemplo, $\false$) y
 \item $S \vdash c$ (mediante resolución).
 \end{itemize}
 \vspace{5mm}

 Entonces, existe una deducción de $c$ en la que
 \begin{itemize}
 \item se parte de una cláusula negativa de $S$,
 \item cada cláusula nueva es negativa y
 \item cada cláusula nueva es la resolvente de la anterior
   ($-$) de la deducción y de una cláusula ($+$) de $S$.
 \end{itemize}

\end{frame}

\begin{frame}
 \frametitle{Resolución con cláusulas de Horn}

 Por ejemplo, para demostrar $\fun{Mortal}(\fun s)$ a partir
 de $\forall x\ (\fun{Hombre}(x) \rightarrow
 \fun{Mortal}(x))$ y $\fun{Hombre}(\fun s)$:
\vspace{1cm}

 \begin{tikzpicture}{>=stealth}
 \node[caja,color=naranja_muy] at (-1, 7) {$S$};
 \node[caja_b] (1) at (0, 6)
      {$\neg \fun{Mortal}(\fun s)\quad (-)$};
 \node[caja_b] (2) at (0, 5)
      {$\neg\fun{Hombre}(x) \lor \fun{Mortal}(x)\quad (+)$};
 \node[caja_b] (3) at (0, 4)
      {$\fun{Hombre}(\fun s)\quad (+)$};
 \draw[color=naranja_muy] (.5,3.5) -- (.5,6.5);
 \node[caja_b] (a1) at (4.5, 5)
      {$\neg \fun{Hombre}(\fun s)\quad (-)$};
 \node[caja_b] (a2) at (6.2, 4)
      {$\false\quad (-)$};
 \draw[->] (1) -| (a1.north west);
 \draw[->] (2) -- (a1);
 \draw[->] (a1) -| (a2.north west);
 \draw[->] (3) -- (a2);
 \end{tikzpicture}
\end{frame}

%% OTRO EJEMPLO ALGO MÁS COMPLEJO

\begin{frame}
 \frametitle{Resolución con cláusulas de Horn}

 Este tipo de resolución se conoce como {\bf SLD}
 (\emph{selected literals, linear pattern, over definite
   clauses}).
 \vspace{3mm}

 {\small Lo describió {\sc R. Kowalski} en 1974 y {\sc
     K. Apt} y {\sc M. van Emden} lo denominaron así en
   1982.}
 \vspace{5mm}

 \pause

 Según lo anteriormente enunciado, el método de resolución
 puede restringirse a la resolución SLD si todas las
 cláusulas de partida son de Horn.
 \vspace{5mm}

 \pause

 En el caso proposicional, en cada resolución se «gasta» una
 literal positiva del conjunto de partida.

 En consecuencia, la longitud de una resolución SLD es
 lineal en el tamaño del conjunto de partida {\small (en
   literales, que no en cláusulas)}.
\end{frame}

\begin{frame}[t]
 \frametitle{Resolución con cláusulas de Horn}

 \only<-2>{
 En el caso general del lenguaje de primer orden, la
 resolución SLD tampoco garantiza finalización. \vspace{3mm}}
 \only<3->{\fbox{\parbox{\linewidth}{
 La restricción del lenguaje de primer orden a las cláusulas
 de Horn sigue siendo {\bf indecidible}
 (semidecidible).}} \vspace{1mm}}

 Por ejemplo, supongamos una \emph{base de conocimiento}
 formada por una única cláusula:
 \[\fun{Apellida}(\fun{padre}(x), y) \rightarrow
   \fun{Apellida}(x, y).\]

 Si tratamos de averiguar si \underline{\fun{pedro}} se
 apellida \underline{\fun{pérez}},
 \only<2->{
 \hypertarget{apellidos}{}
 \hyperlink{ejemplos}{\beamerreturnbutton{Índice}}}
 \vspace{5mm}

\pause

 \begin{tikzpicture}{>=stealth}
 \node[caja,color=naranja_muy] at (6, 6) {$S$};
 \node[caja_b] (1) at (0, 6)
      {$\neg\fun{Apll}(\fun p(x), y) \lor
       \fun{Apll}(x, y)$};
 \node[caja_d] (2) at (.5, 6)
      {$\neg\fun{Apll}(\fun{pedro}, \fun{pérez})$};
 \draw[color=naranja_muy] (-5,5.5) -- (5,5.5);
 \node[caja_d] (a1) at (.5, 5)
      {$\neg\fun{Apll}(\fun{p}(\fun{pedro}),
                           \fun{pérez})$};
 \node[caja_d] (a2) at (.5, 4)
      {$\neg\fun{Apll}(\fun{p}^2(\fun{pedro}),
                           \fun{pérez})$};
 \node[caja_d] (a3) at (.5, 3)
      {$\neg\fun{Apll}(\fun{p}^3(\fun{pedro}),
                           \fun{pérez})$};
 \node at (a3.south west) {$\vdots$};
 %% \node[caja_b] (a4) at (4, 2)
 %%      {$\neg\fun{Apll}(\fun{p}^4(\fun{pedro}),
 %%                           \fun{pérez})$};
 %% \node[caja_b] (a5) at (4, 1)
 %%      {$\neg\fun{Apll}(\fun{p}^5(\fun{pedro}),
 %%                           \fun{pérez})$};
 \draw[->] (1.south) -- (a1.west);
 \draw[->] (1.south) -- (a2.west);
 \draw[->] (1.south) -- (a3.west);
 %% \draw[->] (1.south) -- (a4.west);
 %% \draw[->] (1.south) -- (a5.west);
 \draw[->] (2.south west) -- (a1.north west);
 \draw[->] (a1.south west) -- (a2.north west);
 \draw[->] (a2.south west) -- (a3.north west);
 %% \draw[->] (a3) -- (a4);
 %% \draw[->] (a4) -- (a5);

 \node at (-4, 3) {\parbox{4cm}{nos enredamos en una
     recursión infinita.}};
 \end{tikzpicture}

\end{frame}

\begin{frame}
 \frametitle{Cláusulas de Horn}

 A la vista de los ejemplos, podemos apuntar esta
 clasificación de las cláusulas de Horn.
 \vspace{5mm}

 {\renewcommand\arraystretch{2}
 \begin{tabular}{|rl|rl|}
 \multicolumn{2}{c}{$+$} & \multicolumn{2}{c}{$-$}\\ \hline
 $\sa$ & \emph{fact} & $\neg\sa$ & \emph{query}\\ \hline
 $\neg\sb\ \lor \cdots \lor \neg\sb\ \lor\ \sa$ &
 \emph{rule} &
 $\neg\sa\ \lor \cdots \lor \neg\sa$ & \emph{multiple query}\\
 $(\sb\ \land \cdots \land \sb)\ \rightarrow\ \sa$ & &
 $\neg(\sa\ \land \cdots \land\ \sa)$ & \\ \hline
 \end{tabular}}
\end{frame}

\begin{frame}[t]
 \frametitle{\emph{Goal trees}}

 Además de la notación para las cláusulas que venimos
 utilizando, donde una consulta se traduce en una literal
 negativa, el mecanismo de resolución SLD puede
 representarse como un árbol de objetivos:

 \begin{columns}
 \column{.65\linewidth}
 \begin{tikzpicture}{>=stealth}
 \node[caja,color=naranja_muy] at (-1, 7) {\small $S$};
 \node[caja_b] (1) at (0, 6)
      {\small $(-)\quad\neg \fun{Mortal}(\fun s)$};
 \node[caja_b] (2) at (0, 5)
      {\small $\neg\fun{Hombre}(x) \lor \fun{Mortal}(x)$};
 \node[caja_b] (3) at (0, 4)
      {\small $\fun{Hombre}(\fun s)$};
 \draw[color=naranja_muy] (0.2,3.5) -- (0.2,6.5);
 \node (a1) at (1.5, 5)
      {\small $\neg \fun{Hombre}(\fun s)$};
 \node (a2) at (1.5, 4)
      {\small $\false$};
 \draw[->] (1) -| (a1);
 \draw[->] (2) -- (a1);
 \draw[->] (a1) -- (a2);
 \draw[->] (3) -- (a2);
 \end{tikzpicture}
 \column{.3\linewidth}
 \vspace{7mm}

 \begin{tikzpicture}{>=stealth}
 \node (1) at (0, 6)
      {\small $\fun{Mortal}(\fun s)$};
 \node (2) at (0, 5)
      {\small $\fun{Hombre}(\fun s)$};
 \node (3) at (0, 4)
      {\small $\checkmark$};
 \draw[->] (1) -- (2);
 \draw[->] (2) -- (3);
 \end{tikzpicture}
 \end{columns}

 \hyperlink{ejemplos}{\beamerreturnbutton{Índice}}
 \hypertarget{socrates}{}
\end{frame}

\begin{frame}[t]
 \frametitle{\emph{Goal trees}}

 Además de la notación para las cláusulas que venimos
 utilizando, donde una consulta se traduce en una literal
 negativa, el mecanismo de resolución SLD puede
 representarse como un árbol de objetivos:

 \begin{columns}
 \column{.65\linewidth}
 \begin{tikzpicture}{>=stealth}
 % \node[caja,color=naranja_muy] at (-1, 7) {\small $S$};
 \node[caja_b] (1) at (0, 6)
      {\small $(-)\quad\neg \fun{jabón}$};
 \node[caja_b] (2) at (0, 5)
      {\small $\neg\fun{sebo} \lor \neg\fun{KOH} \lor
               \fun{jabón}$};
 \node[caja_b] (3) at (0, 4)
      {\small $\neg\fun{ceniza} \lor \neg\fun{agua} \lor
               \fun{KOH}$};
 \node[caja_b] (4) at (0, 3)
      {\small $\fun{agua}$};
 \node[caja_b] (5) at (0, 2)
      {\small $\fun{sebo}$};
 \node[caja_b] (6) at (0, 1)
      {\small $\fun{ceniza}$};
 \draw[color=naranja_muy] (0.2,.5) -- (0.2,6.5);
 \node[caja_d] (a1) at (.5, 5)
      {\small $\neg\fun{sebo} \lor \neg\fun{KOH}$};
 \node[caja_d] (a2) at (.5, 4)
      {\small $\neg\fun{sebo} \lor \neg\fun{ceniza} \lor
               \neg\fun{agua}$};
 \node[caja_d] (a3) at (.5, 3)
      {\small $\neg\fun{sebo} \lor \neg\fun{ceniza}$};
 \node[caja_d] (a4) at (.5, 2)
      {\small $\neg\fun{ceniza}$};
 \node[caja_d] (a5) at (.5, 1)
      {\small $\false$};
 \draw[->] (1) -| (a1.north west);
 \draw[->] (2) -- (a1);
 \draw[->] (a1.south west) -- (a2.north west);
 \draw[->] (3) -- (a2);
 \draw[->] (a2.south west) -- (a3.north west);
 \draw[->] (4) -- (a3);
 \draw[->] (a3.south west) -- (a4.north west);
 \draw[->] (5) -- (a4);
 \draw[->] (a4.south west) -- (a5.north west);
 \draw[->] (6) -- (a5);
 \end{tikzpicture}
 \column{.3\linewidth}
 % \vspace{5mm}

 \begin{tikzpicture}{>=stealth}
 \node (0) at (0, 5) {\phantom{a}};
 \node (1) at (2, 7)
      {\small $\fun{jabón}$};
 \node (2) at (1.3, 5)
      {\small $\fun{sebo}$};
 \node (3) at (2.5, 5)
      {\small $\fun{potasa}$};
 \node (4) at (1.5, 3)
      {\small $\fun{ceniza}$};
 \node (5) at (3, 3)
      {\small $\fun{agua}$};
 \node (a2) at (1.3, 4.5) {$\checkmark$};
 \node (a4) at (1.5, 2.5) {$\checkmark$};
 \node (a5) at (3, 2.5) {$\checkmark$};
 \draw[->] (1) -- (2);
 \draw[->] (1) -- (3);
 \draw[->] (3) -- (4);
 \draw[->] (3) -- (5);
 \end{tikzpicture}
 \end{columns}

 \hyperlink{ejemplos}{\beamerreturnbutton{Índice}}
 \hypertarget{sebo}{}
\end{frame}

\def\casa{\mbox{\Large\fontspec{Symbola}🏠}}
\def\bus{\mbox{\Large\fontspec{Symbola}🚌}}
\def\taxi{\mbox{\Large\fontspec{Symbola}🚕}}
\def\pata{\mbox{\Large\fontspec{Symbola}🚶}}
\def\bici{\mbox{\Large\fontspec{Symbola}🚲}}

\begin{frame}[t]
 \frametitle{\emph{Goal trees}}

 Además de la notación para las cláusulas que venimos
 utilizando, donde una consulta se traduce en una literal
 negativa, el mecanismo de resolución SLD puede
 representarse como un árbol de objetivos:

 \begin{columns}
 \column{.65\linewidth}
 \begin{tikzpicture}{>=stealth}
 % \node[caja,color=naranja_muy] at (-1, 7) {\small $S$};
 \node[caja_b] (1) at (0, 6)
      {$(-)\quad\neg\casa$};
 \node[caja_b] (2) at (0, 5)
      {$\neg\bus \lor \casa$};
 \node[caja_b] (3) at (0, 4)
      {$\neg\taxi \lor \casa$};
 \node[caja_b] (4) at (0, 3)
      {$\neg\pata \lor \casa$};
 \node[caja_b] (5) at (0, 2)
      {$\neg\bici \lor \casa$};
 \node[caja_b] (5) at (0, 1)
      {$\bus$};

 \draw[color=naranja_muy] (0.2,.5) -- (0.2,6.5);
 \node (a1) at (1.5, 5)
      {\small $\neg\bus$};
 \node (a2) at (1.5, 1)
      {\small $\false$};
 \draw[->] (1) -| (a1);
 \draw[->] (2) -- (a1);
 \draw[->] (a1) -- (a2);
 \draw[->] (5) -- (a2);
 \end{tikzpicture}
 \column{.3\linewidth}
 % \vspace{7mm}

 \begin{tikzpicture}{>=stealth}
 \node (1a) at (0, 6)
      {$\casa$};
 \node (2a) at (0, 5)
      {$\bus$};
 \node (3a) at (0, 4)
      {\small $\checkmark$};
 \node (1b) at (2, 6)
      {$\casa$};
 \node (2b) at (2, 5)
      {$\taxi$};
 \node (1c) at (0, 2)
      {$\casa$};
 \node (2c) at (0, 1)
      {$\pata$};
 \node (1d) at (2, 2)
      {$\casa$};
 \node (2d) at (2, 1)
      {$\bici$};

 \draw[->] (1a) -- (2a);
 \draw[->] (2a) -- (3a);
 \draw[->] (1b) -- (2b);
 \draw[->] (1c) -- (2c);
 \draw[->] (1d) -- (2d);
 \end{tikzpicture}
 \end{columns}

 \hyperlink{ejemplos}{\beamerreturnbutton{Índice}}
 \hypertarget{casa}{}
\end{frame}

\begin{frame}
 \frametitle{Restricción}

 Al restringir la sentencias lógicas que manejamos, perdemos
 la capacidad de tratar deducciones sencillas como esta:

 \[\begin{array}{rcl}
   p & \rightarrow & q\\
   \neg p & \rightarrow & q\\ \hline
   && q
 \end{array}\]
\vspace{5mm}

 La segunda implicación no es una cláusula de Horn.
\end{frame}

\begin{frame}[t]
 \frametitle{Implicaciones direccionales}

 \only<-2>{
 Las tres sentencias siguientes son semánticamente
 equivalentes: dicen lo mismo con formas distintas.

 \[\begin{array}{rcl}
   p & \rightarrow & q\\
   \neg p & \lor & q\\
   \neg q & \rightarrow & \neg p
 \end{array}\]}

 \only<2>{
 Una estrategia para enfocar el uso computacional de una
 implicación consiste en «direccionarla»,\\
 \hfill\parbox{.9\linewidth}{admitiendo una
 sola de sus dos vías de utilización.}}

 \only<3>{
 En el contexto de los sistemas expertos, se utiliza el
 símbolo $\Rightarrow$ para indicar esa direccionalidad:
 \[\fun{Hombre}(x) \Rightarrow \fun{Mortal}(x).\]
 }

 \only<2->{
 \begin{textblock}{12.5}(1.2,11)
 Así, por ejemplo, la implicación {\color{naranja_muy}
   $\fun{Hombre}(x) \rightarrow \fun{Mortal}(x)$} se
 utilizaría para deducir
 {\color{naranja_muy}$\fun{Mortal}(\fun a)$} a partir de
 {\color{naranja_muy}$\fun{Hombre}(\fun a)$},\\[1mm]
 \hfill\parbox{.9\linewidth}{dejando de lado su potencial
   aplicación para deducir que alguien no es hombre si se
   observa que sobrevive a una caída de doce pisos.}
 \end{textblock}}
\end{frame}

\begin{frame}[t]
 \frametitle{\emph{Backward chaining}}

 En los últimos ejemplos,
 \begin{itemize}
 \item
 Se parte de una consulta (\emph{query}) u objetivo:
 \[\fun{Mortal}(s),\quad
  \fun{Apellida}(\fun{pedro}, \fun{pérez}).\]
 \item
 Esta consulta se traduce en una cláusula negativa de la que
 arranca la resolución SLD.
 \only<2->{
 \item
 Las cláusulas positivas de la base de conocimiento se
 emplean como implicaciones direccionales ({\bf reglas}):
 \[\neg\sa \lor \cdots \lor \neg\sa \lor q\qquad\qquad
 \big(\sa \land \cdots \land \sa\big) \Rightarrow q.\]
 Para demostrar el objetivo (\emph{goal}) $q$, se
 intentan resolver todos los objetivos $\sa,\ldots,\sa$.}
 \only<3->{
 \item
 Un objetivo queda resuelto si viene afirmado por un
 {\bf\emph{fact}} de la base de conocimiento ($\sa$, que es
 $\true\rightarrow\sa$).}
 \end{itemize}

  %% {\small Para demostrar $\fun{Mortal}(s)$, bastará tener
  %% $\fun{Hombre}(s)$.\vspace{2mm}

  %% Para demostrar $\fun{Apll}(\fun{pedro}, \fun{pérez})$,
  %% bastará tener $\fun{Apll}(\fun p(\fun{pedro}),
  %% \fun{pérez})$.}

\end{frame}

\begin{frame}
 \frametitle{\emph{Backward chaining}}

 Este esquema avanza desde la consecuencia de una
 implicación direccional hacia sus
 antecedentes,\\ \hfill\parbox{.9\linewidth}{\small
   retrotrayéndose hasta conseguir (o no) demostrarla.}
 \vspace{3mm}

 Se denomina {\bf backward chaining}.
 \vspace{5mm}

 \pause

 Típicamente, se corresponde con un sistema que toma un
 objetivo como entrada (\emph{goal-directed}), tratando de
 encontrar en la base de conocimiento datos
 (\emph{facts}) que lo sostengan.
 \vspace{5mm}

 %% \pause

 %% {\small También en el caso típico,} desencadena una {\bf
 %%   búsqueda en profundidad} en la base de conocimiento.
\end{frame}

\begin{frame}
 \frametitle{Prolog}

 Prolog es un lenguaje de programación basado en la
 resolución SLD y \emph{backward chaining}:
 \vspace{5mm}

 \begin{itemize}
 \item
 Creado por % Kowalski
 {\sc A. Colmerauer} y {\sc Ph. Roussel} en 1972.
 \item
 Utiliza la {\bf programación lógica} (un tipo de
 programación declarativa).\vspace{2mm}

 \hfill\parbox{.9\linewidth}
 {\small No se trata de un lenguaje declarativo \emph{puro}:
   es necesario tener en cuenta su comportamiento
   procedimental.}\vspace{3mm}

 \item
 La especificación de la tarea determina el mecanismo para
 llevarla a cabo.\vspace{2mm}

  \hfill\parbox{.9\linewidth}
 {\small Aunque distintas maneras de escribir la misma
   especificación lógica dan lugar a resultados radicalmente
   distintos.}
 \end{itemize}
\end{frame}

\begin{frame}
 \frametitle{Prolog}
 Un programa consiste en una serie de {\bf reglas}:

 \[\texttt{Head} \Leftarrow \texttt{body}.\]

 La cabeza se interpreta como un objetivo que se alcanzará
 si se pueden verificar los requisitos del otro término.

 Además, en el programa se pueden establecer
 {\bf\emph{facts}}:

 \begin{center}
 \begin{minipage}[t]{.42\linewidth}
 \VerbatimInput
 %% [frame=lines, label=01\_socrates.py, framesep=4mm,
 %% fontsize=\small]
 {\cod/01_socrates.pl}
 \end{minipage}
 \end{center}

 La última línea equivale a

 % \begin{center}
 % \begin{minipage}[t]{.42\linewidth}
 % \VerbatimInput
 % %% [frame=lines, label=01\_socrates.py, framesep=4mm,
 % %% fontsize=\small]
 % {01_socrates_bis.pl}
 % \end{minipage}
 % \end{center}
\end{frame}

\begin{frame}
 \frametitle{Prolog}

 \hypertarget{ejemplos}{}
 Algunos ejemplos:
 \vspace{5mm}

 \hspace{2cm}
 \begin{tabular}{rl}
 \hyperlink{socrates}{\beamergotobutton{Sócrates es mortal}}\\
 \hyperlink{sebo}{\beamergotobutton{Saponificación}}\\
 \hyperlink{apellidos}{\beamergotobutton{Patronímico}}&
 $\longrightarrow$ \small orden de las líneas del código\\
 \hyperlink{casa}{\beamergotobutton{Volver a casa}}\\
 \end{tabular}
\end{frame}

\begin{frame}
 \frametitle{\emph{Backward chaining}}

 Opera de esta manera el sistema {\bf GNU make}, que se
 emplea para compilar programas con una estructura compleja.
 \vspace{5mm}

 Los ficheros {\bf \texttt{Makefile}} están formados por
 \emph{reglas} que indican cómo generar un fichero objetivo
 (\emph{target}) y de qué ficheros (\emph{prerequisites})
 depende.

 \VerbatimInput{\cod/01_Makefile}

 Así, cuando se le pide construir o reconstruir cierto
 \emph{target}, \emph{make} comprueba sus
 prerrequisitos. Cada uno de ellos se construye,
 recursivamente, si falta o su fecha de modificación es
 anterior a la de alguno de los ficheros de los que depende.
\end{frame}

\begin{frame}
 \frametitle{\emph{Forward chaining}}

 Se conoce como \emph{forward chaining} la estrategia
 opuesta al \emph{backward chaining}.
 \vspace{3mm}

 Se procede desde los \emph{asertos} conocidos hacia sus
 consecuencias, siguiendo el sentido de las implicaciones
 direccionales: {\bf \emph{data-directed}} en vez de
 \emph{goal-directed}.
 \vspace{3mm}

 Así, en vez de obtener objetivos o hipótesis intermedias,
 se trabaja con conclusiones intermedias.
 \vspace{5mm}

 \pause

 Este esquema no se corresponde con el método de resolución.
 \vspace{3mm}

 En particular, no persigue un objetivo concreto, sino que
 va calculando todo lo que se deriva de los datos de
 partida.
 %% \vspace{3mm}

 %% Puede configurarse como una {\bf búsqueda en anchura}.
\end{frame}

\begin{frame}
\centerline{\emph{Sistemas de producción}}
\end{frame}

\begin{frame}
 \frametitle{Sistemas de producción}

 {\bf Sistema experto:} trata de emular a un experto humano.
 \vspace{4mm}

 Se ha tratado [{\sc A. Newell} y {\sc H. Simon}, 1972] de
 modelizar el proceso de una persona para obtener
 conclusiones como\\[2mm] \hfill\parbox{.85\linewidth}{la
   aplicación de múltiples reglas sencillas, que forman su
   {\bf base de conocimiento}.}
 \[\mbox{se oye un ladrido}
   \ \Rightarrow\ 
   \mbox{debe de ser un perro.}\]

 \pause

 Un {\bf sistema de producción} es un sistema experto
 basado en reglas de producción.
\end{frame}

\begin{frame}[t]
 \frametitle{Sistemas de producción}

 Al estudiar la derivación SLD y Prolog,\\[2mm]
 \hfill\parbox{.85\linewidth}{hemos reunido con un
   tratamiento uniforme los conceptos de \emph{facts} y
   \emph{rules}.}
 \vspace{4mm}

 Ahora hacemos una distinción marcada entre
 \begin{itemize}
 \item
 la {\bf base de conocimiento} compuesta por las reglas con
 las que contamos y
 \item
 la {\bf \emph{working memory}} (volátil) con los
 \emph{facts} que manejamos en un determinado momento.
 \end{itemize}
 \vspace{4mm}

 \pause

 Un sistema de producción opera mediante \emph{forward
   chaining},\\[2mm]
 \hfill\parbox{.85\linewidth}{aplicando las reglas que
   disparen los \emph{facts},}\\[2mm]
 \hfill\parbox{.7\linewidth}{hasta que ya no pueda aplicarse
   ninguna.}\\[2mm]
\end{frame}

\begin{frame}
 \frametitle{Sistemas de producción}

 Esta estructura se ajusta al modelo cognitivo de {\sc
   A. Newell} y {\sc H. Simon}{\small, utilizando reglas de
 producción para representar el conocimiento.}
 \vspace{7mm}

 {\renewcommand\arraystretch{1.5}
 \begin{minipage}[t]{.35\linewidth}
 \begin{tabular}{|l|} \hline
 memoria a largo plazo  \\ \hline
 memoria a corto plazo  \\ \hline
 procesador cognitivo  \\ \hline
 \end{tabular}
 \end{minipage}\hfill
 \begin{minipage}[t]{.6\linewidth}
 \begin{tabular}{|ll|} \hline
 base de conocimiento & (reglas) \\ \hline
 \emph{working memory} & (datos) \\ \hline
 motor de inferencia & \\ \hline
 \end{tabular}
 \end{minipage}}
\end{frame}

\begin{frame}[t]
 \frametitle{Ejemplo\hfill{\fontspec{FreeSerif}Ⅰ}}

 \begin{minipage}[t]{.42\linewidth}
 \VerbatimInput{\cod/01_socrates_01.clp}
 \end{minipage}\hfill\vline\hfill
 \begin{minipage}[t]{.55\linewidth}
 \VerbatimInput[commandchars=\\\{\}]{\cod/01_socrates_02.txt}
 \end{minipage}
%%  \vspace{4mm}
%% \pause

%%  La regla se ejecuta una sola vez, aunque la condición que
%%  la dispara sigue presente:\\[1mm]
%%  \hfill\parbox{.85\linewidth}{para evitar bucles, las reglas
%%    son \emph{refractarias} (\emph{refractory}).}
\end{frame}

\begin{frame}
 \frametitle{CLIPS}

 El ejemplo anterior está escrito en {\bf CLIPS} (\emph{C
   Language Integrated Production System}),\\[2mm]
 \hfill\parbox{.85\linewidth}{\small una herramienta de
   desarrollo de sistemas expertos.}
 \vspace{5mm}

 Su sintaxis se basa en la de {\bf LISP} ({\small lenguaje
   de programación funcional por antonomasia}), que tiene
 gran ascendiente sobre la inteligencia artificial.
\end{frame}

\begin{frame}[t]
 \frametitle{Ejemplo\hfill{\fontspec{FreeSerif}Ⅰ}}

 \begin{minipage}[t]{.42\linewidth}
 \VerbatimInput{\cod/01_socrates_01.clp}
 \end{minipage}\hfill\vline\hfill
 \begin{minipage}[t]{.55\linewidth}
 \VerbatimInput[commandchars=\\\{\}]{\cod/01_socrates_02.txt}
 \end{minipage}
 \vspace{4mm}

 La regla se ejecuta una sola vez, aunque la condición que
 la dispara sigue presente:\\[1mm]
 \hfill\parbox{.85\linewidth}{para evitar bucles, las reglas
   son \emph{refractarias} (\emph{refractory}).}
\end{frame}

\begin{frame}[t]
 \frametitle{Ejemplo\hfill{\fontspec{FreeSerif}Ⅱ}}

 \begin{minipage}[t]{.55\linewidth}
 \trucavuelve
 \VerbatimInput{\cod/01_grafo_09.clp}
 \end{minipage}\hfill % \vline\hfill
 \begin{minipage}[t]{.42\linewidth}
 \truca
 \includegraphics{\figs/01_fig_00\sufijo}
 \end{minipage}
 \vspace{5mm}

 La \emph{refracción} no impide que una regla se dispare
 varias veces, en base a distintos \emph{facts}.
 \vspace{2mm}
 \pause

 {\small Para comprobar si una pareja concreta de nodos
   están conectados, es más conveniente el método de
   \emph{backward chaining} que hemos visto.}
\end{frame}

\begin{frame}[t]
 \frametitle{Ejemplo\hfill{\fontspec{FreeSerif}Ⅲ}}

 \begin{minipage}[t]{.55\linewidth}
 \trucavuelve
 \VerbatimInput{\cod/01_grafo_10.clp}
 \end{minipage}\hfill % \vline\hfill
 \begin{minipage}[t]{.42\linewidth}
 \truca
 \includegraphics{\figs/01_fig_00\sufijo}
 \end{minipage}
 \vspace{5mm}

 \only<1>{
 Este sistema de \emph{forward chaining}, en cambio, resulta
 muy conveniente para calcular todos los vértices de una
 componente conexa.}
 \only<2>{
 Si las dos condiciones del antecedente estuvieran en el orden inverso, el proceso sería menos eficiente.}
\end{frame}

\begin{frame}
 \frametitle{Reglas de producción}

 Una {\bf regla de producción} (o simplemente {\bf
   producción}) consta de
 \begin{itemize}
 \item \underline{Antecedente}\hfill
  (\emph{left-hand side (LHS)}, terminología de CLIPS)\\[2mm]

 Sucesión de elementos condicionales,\\[1mm]
 \hfill\parbox{.85\linewidth}{que se combinan mediante {\bf
     conjunción}.}\\[2mm]

 El elemento condicional típico es\\[1mm]
 \hfill\parbox{.85\linewidth}{un patrón ({\bf \emph{pattern}})
   al que debe amoldarse algún \emph{fact}.}
 \vspace{5mm}

 \item \underline{Consecuente}\hfill
  (\emph{right-hand side (RHS)}, terminología de CLIPS)\\[2mm]

 Sucesión de \emph{acciones}.
 \end{itemize}
\end{frame}

\begin{frame}
\centerline{\emph{Sistemas de producción}}
\end{frame}

\begin{frame}
 \frametitle{Sistemas de producción}

 {\bf Sistema experto:} trata de emular a un experto humano.
 \vspace{4mm}

 Se ha tratado [{\sc A. Newell} y {\sc H. Simon}, 1972] de
 modelizar el proceso de una persona para obtener
 conclusiones como\\[2mm] \hfill\parbox{.85\linewidth}{la
   aplicación de múltiples reglas sencillas, que forman su
   {\bf base de conocimiento}.}
 \[\mbox{se oye un ladrido}
   \ \Rightarrow\ 
   \mbox{debe de ser un perro.}\]

 \pause

 Un {\bf sistema de producción} es un sistema experto
 basado en reglas de producción.
\end{frame}

\begin{frame}[t]
 \frametitle{Sistemas de producción}

 Al estudiar la derivación SLD y Prolog,\\[2mm]
 \hfill\parbox{.85\linewidth}{hemos reunido con un
   tratamiento uniforme los conceptos de \emph{facts} y
   \emph{rules}.}
 \vspace{4mm}

 Ahora hacemos una distinción marcada entre
 \begin{itemize}
 \item
 la {\bf base de conocimiento} compuesta por las reglas con
 las que contamos y
 \item
 la {\bf \emph{working memory}} (volátil) con los
 \emph{facts} que manejamos en un determinado momento.
 \end{itemize}
 \vspace{4mm}

 \pause

 Un sistema de producción opera mediante \emph{forward
   chaining},\\[2mm]
 \hfill\parbox{.85\linewidth}{aplicando las reglas que
   disparen los \emph{facts},}\\[2mm]
 \hfill\parbox{.7\linewidth}{hasta que ya no pueda aplicarse
   ninguna.}\\[2mm]
\end{frame}

\begin{frame}
 \frametitle{Sistemas de producción}

 Esta estructura se ajusta al modelo cognitivo de {\sc
   A. Newell} y {\sc H. Simon}{\small, utilizando reglas de
 producción para representar el conocimiento.}
 \vspace{7mm}

 {\renewcommand\arraystretch{1.5}
 \begin{minipage}[t]{.35\linewidth}
 \begin{tabular}{|l|} \hline
 memoria a largo plazo  \\ \hline
 memoria a corto plazo  \\ \hline
 procesador cognitivo  \\ \hline
 \end{tabular}
 \end{minipage}\hfill
 \begin{minipage}[t]{.6\linewidth}
 \begin{tabular}{|ll|} \hline
 base de conocimiento & (reglas) \\ \hline
 \emph{working memory} & (datos) \\ \hline
 motor de inferencia & \\ \hline
 \end{tabular}
 \end{minipage}}
\end{frame}

\begin{frame}[t]
 \frametitle{Ejemplo\hfill{\fontspec{FreeSerif}Ⅰ}}

 \begin{minipage}[t]{.42\linewidth}
 \VerbatimInput{\cod/01_socrates_01.clp}
 \end{minipage}\hfill\vline\hfill
 \begin{minipage}[t]{.55\linewidth}
 \VerbatimInput[commandchars=\\\{\}]{\cod/01_socrates_02.txt}
 \end{minipage}
%%  \vspace{4mm}
%% \pause

%%  La regla se ejecuta una sola vez, aunque la condición que
%%  la dispara sigue presente:\\[1mm]
%%  \hfill\parbox{.85\linewidth}{para evitar bucles, las reglas
%%    son \emph{refractarias} (\emph{refractory}).}
\end{frame}

\begin{frame}
 \frametitle{CLIPS}

 El ejemplo anterior está escrito en {\bf CLIPS} (\emph{C
   Language Integrated Production System}),\\[2mm]
 \hfill\parbox{.85\linewidth}{\small una herramienta de
   desarrollo de sistemas expertos.}
 \vspace{5mm}

 Su sintaxis se basa en la de {\bf LISP} ({\small lenguaje
   de programación funcional por antonomasia}), que tiene
 gran ascendiente sobre la inteligencia artificial.
\end{frame}

\begin{frame}[t]
 \frametitle{Ejemplo\hfill{\fontspec{FreeSerif}Ⅰ}}

 \begin{minipage}[t]{.42\linewidth}
 \VerbatimInput{\cod/01_socrates_01.clp}
 \end{minipage}\hfill\vline\hfill
 \begin{minipage}[t]{.55\linewidth}
 \VerbatimInput[commandchars=\\\{\}]{\cod/01_socrates_02.txt}
 \end{minipage}
 \vspace{4mm}

 La regla se ejecuta una sola vez, aunque la condición que
 la dispara sigue presente:\\[1mm]
 \hfill\parbox{.85\linewidth}{para evitar bucles, las reglas
   son \emph{refractarias} (\emph{refractory}).}
\end{frame}

\begin{frame}[t]
 \frametitle{Ejemplo\hfill{\fontspec{FreeSerif}Ⅱ}}

 \begin{minipage}[t]{.55\linewidth}
 \trucavuelve
 \VerbatimInput{\cod/01_grafo_09.clp}
 \end{minipage}\hfill % \vline\hfill
 \begin{minipage}[t]{.42\linewidth}
 \truca
 \includegraphics{\figs/01_fig_00\sufijo}
 \end{minipage}
 \vspace{5mm}

 La \emph{refracción} no impide que una regla se dispare
 varias veces, en base a distintos \emph{facts}.
 \vspace{2mm}
 \pause

 {\small Para comprobar si una pareja concreta de nodos
   están conectados, es más conveniente el método de
   \emph{backward chaining} que hemos visto.}
\end{frame}

\begin{frame}[t]
 \frametitle{Ejemplo\hfill{\fontspec{FreeSerif}Ⅲ}}

 \begin{minipage}[t]{.55\linewidth}
 \trucavuelve
 \VerbatimInput{\cod/01_grafo_10.clp}
 \end{minipage}\hfill % \vline\hfill
 \begin{minipage}[t]{.42\linewidth}
 \truca
 \includegraphics{\figs/01_fig_00\sufijo}
 \end{minipage}
 \vspace{5mm}

 \only<1>{
 Este sistema de \emph{forward chaining}, en cambio, resulta
 muy conveniente para calcular todos los vértices de una
 componente conexa.}
 \only<2>{
 Si las dos condiciones del antecedente estuvieran en el orden inverso, el proceso sería menos eficiente.}
\end{frame}

\begin{frame}
 \frametitle{Reglas de producción}

 Una {\bf regla de producción} (o simplemente {\bf
   producción}) consta de
 \begin{itemize}
 \item \underline{Antecedente}\hfill
  (\emph{left-hand side (LHS)}, terminología de CLIPS)\\[2mm]

 Sucesión de elementos condicionales,\\[1mm]
 \hfill\parbox{.85\linewidth}{que se combinan mediante {\bf
     conjunción}.}\\[2mm]

 El elemento condicional típico es\\[1mm]
 \hfill\parbox{.85\linewidth}{un patrón ({\bf \emph{pattern}})
   al que debe amoldarse algún \emph{fact}.}
 \vspace{5mm}

 \item \underline{Consecuente}\hfill
  (\emph{right-hand side (RHS)}, terminología de CLIPS)\\[2mm]

 Sucesión de \emph{acciones}.
 \end{itemize}
\end{frame}

\begin{frame}
 \frametitle{Reglas de producción}

 Las acciones del consecuente se ejecutan sucesivamente,
 como en programación procedimental.
 \vspace{5mm}

 De acuerdo con la naturaleza dinámica de la \emph{working
   memory}, las siguientes son acciones típicas:

 \begin{itemize}
 \item añadir un \emph{fact},\\
 \hspace{1.5cm}\texttt{(assert (ejemplo))}
 \item eliminarlo,\\
 \hspace{1.5cm}\texttt{(retract (ejemplo))}
 \item o modificarlo.\\
 \hspace{1.5cm}\texttt{(modify <n.º de fact> (campo nuevo\_valor))}
 \end{itemize}
\end{frame}

% \begin{frame}[t]
%  \frametitle{Ejemplo\hfill{\fontspec{FreeSerif}Ⅳ}}

%  \hspace{-1cm}
%  \begin{minipage}[t]{.45\linewidth}
%  \VerbatimInput[commandchars=\\\{\}]{\cod/01_numerico.clp}
%  \end{minipage}\hfill
%  \begin{minipage}[t]{.55\linewidth}
%  \trucavuelve
%  \VerbatimInput[commandchars=\\\{\}, fontsize=\small,
%                 frame=single]
%    {\cod/01_numerico.txt}

%  \VerbatimInput[commandchars=\\\{\}]{\cod/01_numerico_bis.clp}
%  \end{minipage}
%  \vspace{5mm}
% \end{frame}

\begin{frame}
 \frametitle{Sistemas de producción}

 Cada una de las reglas de producción funciona
 independientemente de las demás,\vspace{3mm}

 sobre unos datos (\emph{working memory}) comunes a todas.
 \vspace{1cm}

 Se puede establecer una analogía con un
 {\bf\color{naranja_muy} programa paralelo},\vspace{3mm}

 aunque las reglas se disparan de una en una.
\end{frame}

\begin{frame}
 \frametitle{Sistemas de producción}

 El \underline{ciclo de funcionamiento} de un sistema de
 producción consta de tres pasos:

 \begin{itemize}
 \item {\bf identificar} las reglas aplicables,
 \vspace{1mm}

 \uncover<2->{
 \hfill\parbox{.85\linewidth}
 {Se {\bf activan} las reglas cuyos antecedentes se
   satisfagan (según los \emph{facts} de la \emph{working
     memory}).
 \vspace{3mm}

 CLIPS coloca estas activaciones en la \emph{agenda}.}}
 \vspace{3mm}

 \item escoger una de ellas y
 \vspace{1mm}

 \uncover<3->{
 \hfill\parbox{.85\linewidth}
 {Si hay varias activaciones, se recurre a un mecanismo de
   resolución de conflictos.}}
 \vspace{3mm}

 \item {\bf ejecutarla} (\emph{fire it}).
 \vspace{1mm}

 \uncover<4>{
 \hfill\parbox{.85\linewidth}
 {Esto produce cambios en la \emph{working memory}.
  \vspace{3mm}

  Si se retira o modifica algún \emph{fact} que hubiera
  activado alguna de regla {\small aún no «disparada»},}
  \vspace{1mm}

 \hfill\parbox{.75\linewidth}
 {se elimina de la agenda esa activación.}}
 \end{itemize}
\end{frame}

%% \begin{frame}
%%  \frametitle{Sistemas de producción}

%%  % Programación paralela
%% \end{frame}

\begin{frame}
 \frametitle{\emph{Pattern matching}}

 El \underline{cuello de botella} de un sistema experto se
 localiza en la identificación de las reglas que procede
 activar.
 \vspace{3mm}

 Estos sistemas consideran muchos \emph{facts} y muchas
 reglas a la vez.
 \vspace{3mm}

 En cada paso del ciclo, debe comprobarse
 si alguna combinación nueva de \emph{facts} casa con el
 antecedente de cada una de las reglas.
 \vspace{5mm}

 \pause

 La invención del algoritmo {\color{naranja_muy} \bf Rete}
 [\textsc{Ch. Forgy}, 1982] para el sistema OPS5 marcó un
 hito decisivo para los sistemas de producción.
 \vspace{3mm}

 Este algoritmo aprovecha
 \begin{itemize}
 \item
 que son pocos los cambios en la \emph{working memory} de
 una etapa a la siguiente y
 \vspace{3mm}

 \item
 que reglas distintas pueden compartir condiciones en sus antecedentes.
 \end{itemize}
\end{frame}

\begin{frame}
 \frametitle{Rete}

 La idea básica consiste en llevar un registro, en todo
 momento, del «punto» de satisfacción de los antecedentes de
 las reglas,\\[2mm] \hfill\parbox{.85\linewidth} {en vez de
   comprobar todas las condiciones cada vez que se modifica
   la \emph{working memory}.}
 \vspace{4mm}

 \pause

 Esas condiciones se organizan en una red.
 \vspace{2mm}

 \begin{itemize}
 \item Está compuesta por nodos de dos tipos:
 \begin{itemize}
 \item[\color{naranja_muy}α)] Admiten los «facts» que cumplan cierta condición.
 \item[\color{naranja_muy}β)] Imponen condiciones que involucran a varios «facts».
 \end{itemize}
 \vspace{2mm}

 \item Cada \emph{fact} que se añade a la \emph{working
   memory} recorre la red hasta donde alcanza.
 \vspace{2mm}
 \item Si se modifica alguno, puede que avance\\[1mm]
 \hfill\parbox{.9\linewidth}{\small (o
   que tenga que retroceder por dejar de cumplir alguna
   condición).}
 \end{itemize}

\end{frame}

\begin{frame}
 \frametitle{Rete}

 Valgan como ejemplo estas instrucciones para los
 cantes del guiñote:
 \vspace{5mm}

\def\retea{{\color{naranja_muy}α}\color{frente}}
\def\reteb{{\color{naranja_muy}β}\color{frente}}

 \begin{tikzpicture}[>=stealth]
 \node (a1) at (0,8) {\retea:};
 \node[right of=a1, node distance=2mm, anchor=west] (b1)
        {tenemos un rey};
 \node (a2) at (5,8) {\retea:};
 \node[right of=a2, node distance=2mm, anchor=west] (b2)
        {tenemos una sota};
 \node (a3) at (2.5,6.5) {\reteb:};
 \node[right of=a3, node distance=2mm, anchor=west] (b3)
        {del mismo palo ($X$)};
 \node (a4) at (-1,4.5) {\retea:};
 \node[right of=a4, node distance=2mm, anchor=west] (b4)
        {baza ganada};
 \node (a5) at (7.5, 6) {\retea:};
 \node[right of=a5, node distance=2mm, anchor=west] (b5)
        {pintan $Y$};
 \node (a6) at (2.5,4.5) {\reteb:};
 \node[right of=a6, node distance=2mm, anchor=west] (b6)
        {$X = Y$};
 \node (a7) at (7,4.5) {\reteb:};
 \node[right of=a7, node distance=2mm, anchor=west] (b7)
        {$X \neq Y$};
 \node (r6) at (2.5,3) {REGLA};
 \node[draw, right of=r6, node distance=8mm, anchor=west]
        (br6)
        {\texttt{cantar 40}};
 \node (r7) at (7,3) {REGLA};
 \node[draw, right of=r7, node distance=8mm, anchor=west]
        (br7)
        {\texttt{cantar 20}};

 \draw[->, shorten >=1mm] (a1.south) -- (a3.north);
 \draw[->, shorten >=1mm] (a2.south) -- (a3.north);
 \draw[->, shorten >=1mm] (a3.south) -- (a6.north);
 \draw[->, shorten >=1mm] (a5.south) -- (a6.north);
 \draw[->, shorten >=1mm] (a3.south) -- (a7.north);
 \draw[->, shorten >=1mm] (a5.south) -- (a7.north);
 \draw[->, shorten >=1mm] (a6.south) -- (r6.north);
 \draw[->, shorten >=1mm] (a7.south) -- (r7.north);
 \draw[->, shorten >=1mm] (a4.south) -- (r6.north);
 \draw[->, shorten >=1mm] (a4.south) -- (r7.north);

 \end{tikzpicture}

\end{frame}

\begin{frame}[t]
 \frametitle{\emph{Pattern matching}}

 En cualquier caso, es importante el orden de los
 antecedentes de una regla\only<1>{,\\[2mm]
 \hfill\parbox{.95\linewidth}{ \small como muestra el
   ejemplo de \emph{Expert Systems. Principles and
     Programming}, de {\sc J. Giarratano} y {\sc G. Riley}
   [sec. 11.5].}
 \vspace{5mm}

 \begin{minipage}[t]{.45\linewidth}
 \VerbatimInput[commandchars=\\\{\}]{\cod/01_orden_01.clp}
 \end{minipage}
 \hfill
 \begin{minipage}[t]{.45\linewidth}
 \VerbatimInput[commandchars=\\\{\}]{\cod/01_orden_02.clp}
 \end{minipage}
 \vspace{5mm}

% La presencia en el sistema de
 La segunda regla es mucho más costosa computacionalmente.}%
 \only<2->{.
 \vspace{5mm}

 En general, se pueden sugerir los siguientes criterios:

 \begin{itemize}
 \item Anteponer los patrones más específicos,\\[2mm]
 \hfill\parbox{.8\linewidth}
       {\small como ilustra el ejemplo anterior.}
 \vspace{3mm}

 \item<3-> Posponer los patrones de \emph{facts} volátiles.\\[2mm]
 \hfill\parbox{.8\linewidth}
 {\small Con esto, se consigue reducir los cambios
   frecuentes en las concordancias parciales con los
   antecedentes de las reglas.}
 \vspace{3mm}

 \item<4-> Anteponer los patrones que se ajustan a pocos
   \emph{facts}.
 \end{itemize}
 \vspace{4mm}}
 \uncover<4>{
 Se trata solo de indicaciones generales: estos criterios
 pueden ser contradictorios.}
\end{frame}

\begin{frame}
 \frametitle{Resolución de conflictos}

 En la guía del usuario de CLIPS, de \textsc{J. Giarratano},
 leemos
 \vspace{5mm}

 \hfill\parbox{.85\linewidth}
 {\it Now you might say, “Well, I’ll just design my expert
   system so that only one rule can possibly be activated at
   one time. Then there is no need for conflict
   resolution”. [...]
   \vspace{2mm}

   The bad news is that this success proves that your
   application can be well represented by a sequential
   program.
   \vspace{2mm}

   So you should have coded it in C, Java, or Ada in the
   first place and not bothered writing it as an expert
   system.}
\end{frame}

\begin{frame}
 \frametitle{Resolución de conflictos}

 Una {\bf estructura de control rígida} en un sistema
 experto\\[2mm] \hfill\parbox{.85\linewidth}{(su diseño
   establece muchas relaciones de prioridad de unas reglas
   con respecto a otras),}
 \vspace{1cm}

 puede revelar un programa secuencial subyacente\\[2mm]
 \hfill\parbox{.85\linewidth}{(y la conveniencia de
   recurrir a la programación convencional).}
\end{frame}

\begin{frame}
 \frametitle{Resolución de conflictos}

 \begin{itemize}
 \item
 El término «producción» se encuentra en el sistema de
 reescritura de cadenas formalizado por {\sc E. Post}
 [1943].
 \vspace{3mm}

 Las reglas de este sistema no se organizaban bajo ninguna
 estructura de control o priorización.
 \vspace{7mm}

 \item
 {\sc A. Markov Jr.} [1954] propuso un sistema de producción
 con reglas jerarquizadas linealmente,
 \vspace{3mm}

 \hfill\parbox{.9\linewidth}{\small del mismo modo que
   prioriza Prolog las distintas alternativas para
   satisfacer un objetivo.}
 \end{itemize}
\end{frame}

\begin{frame}
 \frametitle{\emph{Salience}}

 Una manera de decidir qué regla de la agenda ejecutar es\\[2mm] \hfill\parbox{.85\linewidth} {recurrir a una
   prioridad explícita dada por el programador:}
 \vspace{5mm}

 \begin{minipage}{.5\linewidth}
 \VerbatimInput[commandchars=\\\{\}]{\cod/01_salience.clp}
 \end{minipage}
 \vspace{5mm}

 En CLIPS, las reglas tienen \emph{salience} \texttt{0} por
 defecto.
 \vspace{5mm}

 En el ejemplo del cálculo numérico, hemos priorizado la
 regla que termina el proceso.
\end{frame}

\begin{frame}[t]
 \frametitle{Ejemplo\hfill{\fontspec{FreeSerif}Ⅳ}}

 \hspace{-1cm}
 \begin{minipage}[t]{.45\linewidth}
 \VerbatimInput[commandchars=\\\{\}]{\cod/01_numerico.clp}
 \end{minipage}\hfill
 \begin{minipage}[t]{.55\linewidth}
 \trucavuelve
 \VerbatimInput[commandchars=\\\{\}, fontsize=\small,
                frame=single]
   {\cod/01_numerico.txt}

 \VerbatimInput[commandchars=\\\{\}]{\cod/01_numerico_ter.clp}
 \end{minipage}
 \vspace{5mm}
\end{frame}

\begin{frame}[t]
 \frametitle{\emph{Salience}}

 Leemos en el libro \emph{Expert Systems. Principles and
   Programming}, {\sc J. Giarratano} y {\sc G. Riley}
 [p. 453]:
 \vspace{5mm}

 \only<1>{
 \hfill\parbox{.85\linewidth}
 {\it People who are just learning rule-based programming tend
   to overuse salience because it gives them explicit
   control of execution. It is more like the procedural
   programming they are used to [...]
  \vspace{3mm}

  Overuse of salience results in a poorly coded program. The
  main advantage of a rule-based system is that the
  programmer does not have to worry about controlling
  execution.}}

 \only<2>{
 \hfill\parbox{.85\linewidth} {\it Salience should primarily be
   used as a mechanism for determining the order in which
   rules fire. This means that in general a rule that is
   placed on the agenda is eventually fired. Salience should
   not be used as a method for selecting a single rule from
   a group of rules when patterns can be used to express the
   criteria for selection, nor should it be used as a “quick
   fix” to get rules to fire in the proper order.}}
\end{frame}

\begin{frame}
 \frametitle{Resolución de conflictos}

 Si la agenda contiene varias reglas activadas con la misma
 \emph{salience},
 \hfill\parbox{.85\linewidth}
 {CLIPS se decanta por una según el criterio seleccionado:}

 \begin{itemize}
 \item {\bf\color{naranja_muy}\texttt{depth}}

 \begin{minipage}[t]{.6\linewidth}
 %% Es el mecanismo de selección por defecto. El sistema «se
 %% concentra» en lo que está haciendo.
 Opción por defecto. El sistema «se
 concentra» en lo que está haciendo.
 \end{minipage}\hfill\vline\hfill
 \begin{minipage}[t]{.35\linewidth}
 \VerbatimInput[commandchars=\\\{\}]{\cod/01_strategy.txt}
 \end{minipage}

 \item {\bf\color{naranja_muy}\texttt{breadth}}

 Todas las reglas tienen ocasión de intervenir.

 \item {\bf\color{naranja_muy}\texttt{simplicity}}

 \item {\bf\color{naranja_muy}\texttt{complexity}}

 Se priorizan las reglas con antecedente más específico.

 \item {\bf\color{naranja_muy} ···}

 \item {\bf\color{naranja_muy}\texttt{random}}

 Puede ser útil para detectar problemas en el sistema.

 \end{itemize}

 Otro criterio de elección atiende al
 {\bf\color{naranja_muy} orden} en que están definidas las
 reglas (como hace Prolog).
\end{frame}

% \begin{frame}[t]
%  \frametitle{Resolución de conflictos\hfill\texttt{breadth}}

%  \hspace{-1cm}
%  \begin{minipage}[t]{.45\linewidth}
%  \VerbatimInput[commandchars=\\\{\}]{\cod/01_numerico.clp}
%  \end{minipage}\hfill
%  \begin{minipage}[t]{.55\linewidth}
%  \trucavuelve
%  \VerbatimInput[commandchars=\\\{\}, fontsize=\small,
%                 frame=single]
%    {\cod/01_numerico_b.txt}

%  \VerbatimInput[commandchars=\\\{\}]{\cod/01_numerico_quater.clp}
%  \end{minipage}
%  \vspace{5mm}
% \end{frame}

\begin{frame}
 \frametitle{Sistemas de producción}

 Esta manera de programar presenta se caracteriza por

 \begin{itemize}
 \item
 la sencillez de sus estructuras de control y

 \item
 la transparencia de su funcionamiento\\[2mm]

 (comparándola, por ejemplo, con el de las redes neuronales).
 \end{itemize}
 \vspace{5mm}

 Facilita, además del resultado, una {\bf explicación} de la
 cadena de deducciones que ha conducido hasta él.
\end{frame}


\begin{frame}
 \frametitle{Reglas de producción}

 Las acciones del consecuente se ejecutan sucesivamente,
 como en programación procedimental.
 \vspace{5mm}

 De acuerdo con la naturaleza dinámica de la \emph{working
   memory}, las siguientes son acciones típicas:

 \begin{itemize}
 \item añadir un \emph{fact},\\
 \hspace{1.5cm}\texttt{(assert (ejemplo))}
 \item eliminarlo,\\
 \hspace{1.5cm}\texttt{(retract (ejemplo))}
 \item o modificarlo.\\
 \hspace{1.5cm}\texttt{(modify <n.º de fact> (campo nuevo\_valor))}
 \end{itemize}
\end{frame}

% \begin{frame}[t]
%  \frametitle{Ejemplo\hfill{\fontspec{FreeSerif}Ⅳ}}

%  \hspace{-1cm}
%  \begin{minipage}[t]{.45\linewidth}
%  \VerbatimInput[commandchars=\\\{\}]{\cod/01_numerico.clp}
%  \end{minipage}\hfill
%  \begin{minipage}[t]{.55\linewidth}
%  \trucavuelve
%  \VerbatimInput[commandchars=\\\{\}, fontsize=\small,
%                 frame=single]
%    {\cod/01_numerico.txt}

%  \VerbatimInput[commandchars=\\\{\}]{\cod/01_numerico_bis.clp}
%  \end{minipage}
%  \vspace{5mm}
% \end{frame}

\begin{frame}
 \frametitle{Sistemas de producción}

 Cada una de las reglas de producción funciona
 independientemente de las demás,\vspace{3mm}

 sobre unos datos (\emph{working memory}) comunes a todas.
 \vspace{1cm}

 Se puede establecer una analogía con un
 {\bf\color{naranja_muy} programa paralelo},\vspace{3mm}

 aunque las reglas se disparan de una en una.
\end{frame}

\begin{frame}
 \frametitle{Sistemas de producción}

 El \underline{ciclo de funcionamiento} de un sistema de
 producción consta de tres pasos:

 \begin{itemize}
 \item {\bf identificar} las reglas aplicables,
 \vspace{1mm}

 \uncover<2->{
 \hfill\parbox{.85\linewidth}
 {Se {\bf activan} las reglas cuyos antecedentes se
   satisfagan (según los \emph{facts} de la \emph{working
     memory}).
 \vspace{3mm}

 CLIPS coloca estas activaciones en la \emph{agenda}.}}
 \vspace{3mm}

 \item escoger una de ellas y
 \vspace{1mm}

 \uncover<3->{
 \hfill\parbox{.85\linewidth}
 {Si hay varias activaciones, se recurre a un mecanismo de
   resolución de conflictos.}}
 \vspace{3mm}

 \item {\bf ejecutarla} (\emph{fire it}).
 \vspace{1mm}

 \uncover<4>{
 \hfill\parbox{.85\linewidth}
 {Esto produce cambios en la \emph{working memory}.
  \vspace{3mm}

  Si se retira o modifica algún \emph{fact} que hubiera
  activado alguna de regla {\small aún no «disparada»},}
  \vspace{1mm}

 \hfill\parbox{.75\linewidth}
 {se elimina de la agenda esa activación.}}
 \end{itemize}
\end{frame}

%% \begin{frame}
%%  \frametitle{Sistemas de producción}

%%  % Programación paralela
%% \end{frame}

\begin{frame}
 \frametitle{\emph{Pattern matching}}

 El \underline{cuello de botella} de un sistema experto se
 localiza en la identificación de las reglas que procede
 activar.
 \vspace{3mm}

 Estos sistemas consideran muchos \emph{facts} y muchas
 reglas a la vez.
 \vspace{3mm}

 En cada paso del ciclo, debe comprobarse
 si alguna combinación nueva de \emph{facts} casa con el
 antecedente de cada una de las reglas.
 \vspace{5mm}

 \pause

 La invención del algoritmo {\color{naranja_muy} \bf Rete}
 [\textsc{Ch. Forgy}, 1982] para el sistema OPS5 marcó un
 hito decisivo para los sistemas de producción.
 \vspace{3mm}

 Este algoritmo aprovecha
 \begin{itemize}
 \item
 que son pocos los cambios en la \emph{working memory} de
 una etapa a la siguiente y
 \vspace{3mm}

 \item
 que reglas distintas pueden compartir condiciones en sus antecedentes.
 \end{itemize}
\end{frame}

\begin{frame}
 \frametitle{Rete}

 La idea básica consiste en llevar un registro, en todo
 momento, del «punto» de satisfacción de los antecedentes de
 las reglas,\\[2mm] \hfill\parbox{.85\linewidth} {en vez de
   comprobar todas las condiciones cada vez que se modifica
   la \emph{working memory}.}
 \vspace{4mm}

 \pause

 Esas condiciones se organizan en una red.
 \vspace{2mm}

 \begin{itemize}
 \item Está compuesta por nodos de dos tipos:
 \begin{itemize}
 \item[\color{naranja_muy}α)] Admiten los «facts» que cumplan cierta condición.
 \item[\color{naranja_muy}β)] Imponen condiciones que involucran a varios «facts».
 \end{itemize}
 \vspace{2mm}

 \item Cada \emph{fact} que se añade a la \emph{working
   memory} recorre la red hasta donde alcanza.
 \vspace{2mm}
 \item Si se modifica alguno, puede que avance\\[1mm]
 \hfill\parbox{.9\linewidth}{\small (o
   que tenga que retroceder por dejar de cumplir alguna
   condición).}
 \end{itemize}

\end{frame}

\begin{frame}
 \frametitle{Rete}

 Valgan como ejemplo estas instrucciones para los
 cantes del guiñote:
 \vspace{5mm}

\def\retea{{\color{naranja_muy}α}\color{frente}}
\def\reteb{{\color{naranja_muy}β}\color{frente}}

 \begin{tikzpicture}[>=stealth]
 \node (a1) at (0,8) {\retea:};
 \node[right of=a1, node distance=2mm, anchor=west] (b1)
        {tenemos un rey};
 \node (a2) at (5,8) {\retea:};
 \node[right of=a2, node distance=2mm, anchor=west] (b2)
        {tenemos una sota};
 \node (a3) at (2.5,6.5) {\reteb:};
 \node[right of=a3, node distance=2mm, anchor=west] (b3)
        {del mismo palo ($X$)};
 \node (a4) at (-1,4.5) {\retea:};
 \node[right of=a4, node distance=2mm, anchor=west] (b4)
        {baza ganada};
 \node (a5) at (7.5, 6) {\retea:};
 \node[right of=a5, node distance=2mm, anchor=west] (b5)
        {pintan $Y$};
 \node (a6) at (2.5,4.5) {\reteb:};
 \node[right of=a6, node distance=2mm, anchor=west] (b6)
        {$X = Y$};
 \node (a7) at (7,4.5) {\reteb:};
 \node[right of=a7, node distance=2mm, anchor=west] (b7)
        {$X \neq Y$};
 \node (r6) at (2.5,3) {REGLA};
 \node[draw, right of=r6, node distance=8mm, anchor=west]
        (br6)
        {\texttt{cantar 40}};
 \node (r7) at (7,3) {REGLA};
 \node[draw, right of=r7, node distance=8mm, anchor=west]
        (br7)
        {\texttt{cantar 20}};

 \draw[->, shorten >=1mm] (a1.south) -- (a3.north);
 \draw[->, shorten >=1mm] (a2.south) -- (a3.north);
 \draw[->, shorten >=1mm] (a3.south) -- (a6.north);
 \draw[->, shorten >=1mm] (a5.south) -- (a6.north);
 \draw[->, shorten >=1mm] (a3.south) -- (a7.north);
 \draw[->, shorten >=1mm] (a5.south) -- (a7.north);
 \draw[->, shorten >=1mm] (a6.south) -- (r6.north);
 \draw[->, shorten >=1mm] (a7.south) -- (r7.north);
 \draw[->, shorten >=1mm] (a4.south) -- (r6.north);
 \draw[->, shorten >=1mm] (a4.south) -- (r7.north);

 \end{tikzpicture}

\end{frame}

\begin{frame}[t]
 \frametitle{\emph{Pattern matching}}

 En cualquier caso, es importante el orden de los
 antecedentes de una regla\only<1>{,\\[2mm]
 \hfill\parbox{.95\linewidth}{ \small como muestra el
   ejemplo de \emph{Expert Systems. Principles and
     Programming}, de {\sc J. Giarratano} y {\sc G. Riley}
   [sec. 11.5].}
 \vspace{5mm}

 \begin{minipage}[t]{.45\linewidth}
 \VerbatimInput[commandchars=\\\{\}]{\cod/01_orden_01.clp}
 \end{minipage}
 \hfill
 \begin{minipage}[t]{.45\linewidth}
 \VerbatimInput[commandchars=\\\{\}]{\cod/01_orden_02.clp}
 \end{minipage}
 \vspace{5mm}

% La presencia en el sistema de
 La segunda regla es mucho más costosa computacionalmente.}%
 \only<2->{.
 \vspace{5mm}

 En general, se pueden sugerir los siguientes criterios:

 \begin{itemize}
 \item Anteponer los patrones más específicos,\\[2mm]
 \hfill\parbox{.8\linewidth}
       {\small como ilustra el ejemplo anterior.}
 \vspace{3mm}

 \item<3-> Posponer los patrones de \emph{facts} volátiles.\\[2mm]
 \hfill\parbox{.8\linewidth}
 {\small Con esto, se consigue reducir los cambios
   frecuentes en las concordancias parciales con los
   antecedentes de las reglas.}
 \vspace{3mm}

 \item<4-> Anteponer los patrones que se ajustan a pocos
   \emph{facts}.
 \end{itemize}
 \vspace{4mm}}
 \uncover<4>{
 Se trata solo de indicaciones generales: estos criterios
 pueden ser contradictorios.}
\end{frame}

\begin{frame}
 \frametitle{Resolución de conflictos}

 En la guía del usuario de CLIPS, de \textsc{J. Giarratano},
 leemos
 \vspace{5mm}

 \hfill\parbox{.85\linewidth}
 {\it Now you might say, “Well, I’ll just design my expert
   system so that only one rule can possibly be activated at
   one time. Then there is no need for conflict
   resolution”. [...]
   \vspace{2mm}

   The bad news is that this success proves that your
   application can be well represented by a sequential
   program.
   \vspace{2mm}

   So you should have coded it in C, Java, or Ada in the
   first place and not bothered writing it as an expert
   system.}
\end{frame}

\begin{frame}
 \frametitle{Resolución de conflictos}

 Una {\bf estructura de control rígida} en un sistema
 experto\\[2mm] \hfill\parbox{.85\linewidth}{(su diseño
   establece muchas relaciones de prioridad de unas reglas
   con respecto a otras),}
 \vspace{1cm}

 puede revelar un programa secuencial subyacente\\[2mm]
 \hfill\parbox{.85\linewidth}{(y la conveniencia de
   recurrir a la programación convencional).}
\end{frame}

\begin{frame}
 \frametitle{Resolución de conflictos}

 \begin{itemize}
 \item
 El término «producción» se encuentra en el sistema de
 reescritura de cadenas formalizado por {\sc E. Post}
 [1943].
 \vspace{3mm}

 Las reglas de este sistema no se organizaban bajo ninguna
 estructura de control o priorización.
 \vspace{7mm}

 \item
 {\sc A. Markov Jr.} [1954] propuso un sistema de producción
 con reglas jerarquizadas linealmente,
 \vspace{3mm}

 \hfill\parbox{.9\linewidth}{\small del mismo modo que
   prioriza Prolog las distintas alternativas para
   satisfacer un objetivo.}
 \end{itemize}
\end{frame}

\begin{frame}
 \frametitle{\emph{Salience}}

 Una manera de decidir qué regla de la agenda ejecutar es\\[2mm] \hfill\parbox{.85\linewidth} {recurrir a una
   prioridad explícita dada por el programador:}
 \vspace{5mm}

 \begin{minipage}{.5\linewidth}
 \VerbatimInput[commandchars=\\\{\}]{\cod/01_salience.clp}
 \end{minipage}
 \vspace{5mm}

 En CLIPS, las reglas tienen \emph{salience} \texttt{0} por
 defecto.
 \vspace{5mm}

 En el ejemplo del cálculo numérico, hemos priorizado la
 regla que termina el proceso.
\end{frame}

% \begin{frame}[t]
%  \frametitle{Ejemplo\hfill{\fontspec{FreeSerif}Ⅳ}}

%  \hspace{-1cm}
%  \begin{minipage}[t]{.45\linewidth}
%  \VerbatimInput[commandchars=\\\{\}]{\cod/01_numerico.clp}
%  \end{minipage}\hfill
%  \begin{minipage}[t]{.55\linewidth}
%  \trucavuelve
%  \VerbatimInput[commandchars=\\\{\}, fontsize=\small,
%                 frame=single]
%    {\cod/01_numerico.txt}

%  \VerbatimInput[commandchars=\\\{\}]{\cod/01_numerico_ter.clp}
%  \end{minipage}
%  \vspace{5mm}
% \end{frame}

\begin{frame}[t]
 \frametitle{\emph{Salience}}

 Leemos en el libro \emph{Expert Systems. Principles and
   Programming}, {\sc J. Giarratano} y {\sc G. Riley}
 [p. 453]:
 \vspace{5mm}

 \only<1>{
 \hfill\parbox{.85\linewidth}
 {\it People who are just learning rule-based programming tend
   to overuse salience because it gives them explicit
   control of execution. It is more like the procedural
   programming they are used to [...]
  \vspace{3mm}

  Overuse of salience results in a poorly coded program. The
  main advantage of a rule-based system is that the
  programmer does not have to worry about controlling
  execution.}}

 \only<2>{
 \hfill\parbox{.85\linewidth} {\it Salience should primarily be
   used as a mechanism for determining the order in which
   rules fire. This means that in general a rule that is
   placed on the agenda is eventually fired. Salience should
   not be used as a method for selecting a single rule from
   a group of rules when patterns can be used to express the
   criteria for selection, nor should it be used as a “quick
   fix” to get rules to fire in the proper order.}}
\end{frame}

\begin{frame}
 \frametitle{Resolución de conflictos}

 Si la agenda contiene varias reglas activadas con la misma
 \emph{salience},
 \hfill\parbox{.85\linewidth}
 {CLIPS se decanta por una según el criterio seleccionado:}

 \begin{itemize}
 \item {\bf\color{naranja_muy}\texttt{depth}}

 \begin{minipage}[t]{.6\linewidth}
 %% Es el mecanismo de selección por defecto. El sistema «se
 %% concentra» en lo que está haciendo.
 Opción por defecto. El sistema «se
 concentra» en lo que está haciendo.
 \end{minipage}\hfill\vline\hfill
 \begin{minipage}[t]{.35\linewidth}
 \VerbatimInput[commandchars=\\\{\}]{\cod/01_strategy.txt}
 \end{minipage}

 \item {\bf\color{naranja_muy}\texttt{breadth}}

 Todas las reglas tienen ocasión de intervenir.

 \item {\bf\color{naranja_muy}\texttt{simplicity}}

 \item {\bf\color{naranja_muy}\texttt{complexity}}

 Se priorizan las reglas con antecedente más específico.

 \item {\bf\color{naranja_muy} ···}

 \item {\bf\color{naranja_muy}\texttt{random}}

 Puede ser útil para detectar problemas en el sistema.

 \end{itemize}

 Otro criterio de elección atiende al
 {\bf\color{naranja_muy} orden} en que están definidas las
 reglas (como hace Prolog).
\end{frame}

% \begin{frame}[t]
%  \frametitle{Resolución de conflictos\hfill\texttt{breadth}}

%  \hspace{-1cm}
%  \begin{minipage}[t]{.45\linewidth}
%  \VerbatimInput[commandchars=\\\{\}]{\cod/01_numerico.clp}
%  \end{minipage}\hfill
%  \begin{minipage}[t]{.55\linewidth}
%  \trucavuelve
%  \VerbatimInput[commandchars=\\\{\}, fontsize=\small,
%                 frame=single]
%    {\cod/01_numerico_b.txt}

%  \VerbatimInput[commandchars=\\\{\}]{\cod/01_numerico_quater.clp}
%  \end{minipage}
%  \vspace{5mm}
% \end{frame}

\begin{frame}
 \frametitle{Sistemas de producción}

 Esta manera de programar presenta se caracteriza por

 \begin{itemize}
 \item
 la sencillez de sus estructuras de control y

 \item
 la transparencia de su funcionamiento\\[2mm]

 (comparándola, por ejemplo, con el de las redes neuronales).
 \end{itemize}
 \vspace{5mm}

 Facilita, además del resultado, una {\bf explicación} de la
 cadena de deducciones que ha conducido hasta él.
\end{frame}

\end{document}